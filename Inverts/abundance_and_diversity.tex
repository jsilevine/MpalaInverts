% Created 2022-01-20 Thu 10:37
% Intended LaTeX compiler: pdflatex
\documentclass[11pt]{article}
\usepackage[utf8]{inputenc}
\usepackage[T1]{fontenc}
\usepackage{graphicx}
\usepackage{longtable}
\usepackage{wrapfig}
\usepackage{rotating}
\usepackage[normalem]{ulem}
\usepackage{amsmath}
\usepackage{amssymb}
\usepackage{capt-of}
\usepackage{hyperref}
\usepackage{amsmath}
\author{Jacob Levine}
\date{\today}
\title{Multivariate analyses for UHURU Invert data}
\hypersetup{
 pdfauthor={Jacob Levine},
 pdftitle={Multivariate analyses for UHURU Invert data},
 pdfkeywords={},
 pdfsubject={},
 pdfcreator={Emacs 27.2 (Org mode 9.6)}, 
 pdflang={English}}
\begin{document}

\maketitle
\tableofcontents


\section*{0. Prereqs}
\label{sec:orgf4ef10e}

Load packages and plot utility script.

\begin{verbatim}

require(VGAM) ## for fitting multivariate glms
require(ggplot2) ## for plotting
require(cowplot) ## for combining the plots

source("plot_utility.R") ## my personalized plotting theme and functions

\end{verbatim}


\section*{1. Abundance}
\label{sec:org6ada04b}

\subsection*{1.1 Modelling approach}
\label{sec:org3f0d651}

Let's start with the abundance data. The basic idea here is to fit a model which takes into account a possible correlation structure among the responses (order abundance) in order to avoid the typical pitfalls of multiple testing (i.e. fucked type 1 error rates). Because the responses are counts, we should also use some type of Poisson GLM. A nice way to accomplish both these things is to fit a Multivariate Poisson LogNormal Model. The multivariate Poisson LogNormal model is given by the following:

\begin{equation*}
Z_i \sim\ N(\mu_i, \Sigma)
\end{equation*}
\begin{equation}
Y_{i,j} | Z_{i,j} \sim\ P(exp(Z_{i,j}))
\end{equation}
\begin{equation*}
\mu_i = x_i^T\theta_j
\end{equation*}


where \(Y_{i,j}\) is the number of individuals of order \(j\) observed at site \(i\), \(\mu_i\) is the vector of expected log abundances at site \(i\) and is calculated from the covariates, \(x_i\), and estimated coefficients, \(\theta_j\). \(\Sigma\) gives the correlation structure among orders, and \(Z_i\) is referred to as the latent vector. We can fit this class of models using the \texttt{VGAM} package in R.

We are interested primarily in the effect of treatment on order abundance, though also need to control for the experimental design. There are several ways to do this, but the easiest is probably to estimate a fixed effect for each unique replicate (6 in total). This gives us two covariates: Treatment ('C', 'M', or 'T') and Replicate ('1N', '2N', '3N', '1S', '2S', '3S'), which we will model using an interaction between Block and Replicate -- thus giving us three covariates.

\subsection*{1.2 Loading the data}
\label{sec:orgbdec0cb}

Now we can load in some data and get it into a nice format:

\begin{verbatim}

## load data
uhurua <- read.csv("UHURU_summary_metadata_abundance.csv", row.names = 1, header = TRUE)

## remove "abundance" from the column names
removeWords <- function(str, stopwords) {

  x <- unlist(strsplit(str, "_"))
  paste(x[!x %in% stopwords], collapse = " ")

}

colnames(uhurua) <- apply(X = matrix(colnames(uhurua)), MARGIN = 1, FUN = removeWords, stopwords = "abundance")

## inspect data
uhurua[,16:31]

\end{verbatim}

\begin{verbatim}
    Acari Aranae Blattodea Coleoptera Diptera Gatropoda Hemiptera Hymenoptera
N1C     4     28         0         21     128         0       115        1319
N1M     3     35         0         59      61         0       181         486
N1T     4     34         2         39      80         1       147         938
N2C     4     23         0         96      93         0       204        2830
N2M     6     27         1         59     113         0       320        2813
N2T     0     13         0         46      66         0       108        1828
N3C     3     30         0         41      32         0       337         629
N3M     4     43         0         81     107         0       453         845
N3T     1     42         0         28      40         0       126        1028
S1C     5     10         0         56     101         1       482         378
S1M     3     15         0        100     119         0      1248         258
S1T     1     36         0         88     131         0       231         254
S2C    16     17         0         88     226         0       430         349
S2M     7     36         0        176     126         0      1595         425
S2T     3     27         0        117      70         2       236         351
S3C     4     33         0         64      68         1       475         173
S3M     0     27         0         89      42         1       388         662
S3T     3     38         0        109      96         0      1122         168
    Isopoda Lepidoptera Mantodea Neuroptera Orthoptera Phasmatodea Psocoptera
N1C       0          29        5          0         47           7          0
N1M       0          35        1          2         56           6          0
N1T       0          34        0          0         55          23          0
N2C       0          67        0          0         35           7          0
N2M       0          42        0         13         29           8          0
N2T       0          47        0          0         35          22          7
N3C       0          13        0         14        101          15          0
N3M       0          34        0         15         22           8          0
N3T       0           0        2          5         17          27          0
S1C       0          31       10          9         23           6          0
S1M       0          65        2         81         24           3          0
S1T       0          36       10          3         22           8          0
S2C       5          46       25         14         29           8          0
S2M       1         145        2        147         19           4          0
S2T       0          50        7          9         17           3          0
S3C       1          29        5         15         25           5          0
S3M       1          26        6         16         15           3          0
S3T       1          66        8         37         21           8          0
    Solifugae
N1C         6
N1M         0
N1T         0
N2C         0
N2M         0
N2T         1
N3C         0
N3M         0
N3T         0
S1C         0
S1M         0
S1T         0
S2C         0
S2M         0
S2T         0
S3C         0
S3M         0
S3T         0
\end{verbatim}

Some of the species seem to have very sparse data (Solifugae, Psocoptera, Isopoda, Gatropoda, Blattodea). I think it is probably best that we remove these from our analysis (by just not fitting models on them) for the time being as the model fits for them will likely be weak and its unlikely we would glean anything terribly exciting about them anyways.

\subsection*{1.3 Fitting the model}
\label{sec:orgb6e5efa}

\begin{verbatim}

## make sure replicate is factor and not numeric
uhurua$Replicate <- as.factor(uhurua$Replicate)

## fit a vglm
abund_model <- vglm(cbind(Acari,
                          Aranae,
                          Coleoptera,
                          Diptera,
                          Hemiptera,
                          Hymenoptera,
                          Lepidoptera,
                          Mantodea,
                          Neuroptera,
                          Orthoptera,
                          Phasmatodea) ~ Treatment + Block + Block:Replicate,
                    family = "poissonff",
                    data = uhurua)

## extract information we want
summary <- summary(abund_model)
coef_table <- summary@coef3

## utility function to make the names nicer to read
rename <- function(x) {

  split <- unlist(strsplit(x, ":"))
  num <- split[length(split)]
  spp <- colnames(summary@y)[as.numeric(num)]
  if (length(split) > 2) {

    newname <- paste0(spp, ":", split[1], ":", split[2])

  }
  else {

    newname <- paste0(spp, ":", split[1])

  }

  return(newname)

}

## employ our utility function
rownames(coef_table) <- apply(X = matrix(rownames(coef_table), ncol = 1), MARGIN = 1, FUN = rename)

## generate new, easier to read output table
simple_coef <- coef_table[!grepl("Replicate", rownames(coef_table)),]
simple_coef <- data.frame(simple_coef[order(rownames(simple_coef)),])

## add some significance/clarity signifiers
for (i in 1:nrow(simple_coef)) {

  p.value <- simple_coef[i, "Pr...z.."]
  if (p.value < 0.05) clar <- "*"
  else clar <- " "
  simple_coef[i, "clarity"] <- clar

}
colnames(simple_coef) <- c("Estimate", "std.error", "z.value", "p.value", "clarity")

## check out the results
simple_coef[,c(1,2,4,5)]

\end{verbatim}

\begin{verbatim}
                           Estimate  std.error       p.value clarity
Acari:(Intercept)        1.71873433 0.32342288  1.071261e-07       *
Acari:BlockS            -0.20067070 0.44946657  6.552620e-01        
Acari:TreatmentM        -0.44802472 0.26693827  9.327230e-02        
Acari:TreatmentT        -1.09861229 0.33333333  9.812898e-04       *
Aranae:(Intercept)       3.28124760 0.12432197 1.642412e-153       *
Aranae:BlockS           -0.46383711 0.16340967  4.532720e-03       *
Aranae:TreatmentM        0.26072626 0.11205659  1.997924e-02       *
Aranae:TreatmentT        0.29826418 0.11115465  7.289386e-03       *
Coleoptera:(Intercept)   3.46872517 0.10197393 1.300942e-253       *
Coleoptera:BlockS        0.71804473 0.11181110  1.345440e-10       *
Coleoptera:TreatmentM    0.43242092 0.06712146  1.176261e-10       *
Coleoptera:TreatmentT    0.15415068 0.07123314  3.046203e-02       *
Diptera:(Intercept)      4.63080695 0.06835276  0.000000e+00       *
Diptera:BlockS           0.26607484 0.08103379  1.025249e-03       *
Diptera:TreatmentM      -0.13176928 0.05747846  2.187690e-02       *
Diptera:TreatmentT      -0.29387404 0.06011325  1.015179e-06       *
Hemiptera:(Intercept)    4.70409886 0.05123311  0.000000e+00       *
Hemiptera:BlockS         1.48764006 0.05260495 6.181386e-176       *
Hemiptera:TreatmentM     0.71708739 0.02698935 1.538112e-155       *
Hemiptera:TreatmentT    -0.03638577 0.03157674  2.491992e-01        
Hymenoptera:(Intercept)  6.89758257 0.02184322  0.000000e+00       *
Hymenoptera:BlockS      -1.12558603 0.03857667 3.700133e-187       *
Hymenoptera:TreatmentM  -0.03385297 0.01892884  7.370609e-02        
Hymenoptera:TreatmentT  -0.21774252 0.01987663  6.308737e-28       *
Lepidoptera:(Intercept)  3.27726339 0.11660782 8.508956e-174       *
Lepidoptera:BlockS       0.29783444 0.13334106  2.550751e-02       *
Lepidoptera:TreatmentM   0.47868675 0.08679290  3.482339e-08       *
Lepidoptera:TreatmentT   0.08040043 0.09456748  3.952189e-01        
Mantodea:(Intercept)     1.17958135 0.42052430  5.031262e-03       *
Mantodea:BlockS          1.29928298 0.46056619  4.786585e-03       *
Mantodea:TreatmentM     -1.40876722 0.33634988  2.809303e-05       *
Mantodea:TreatmentT     -0.51082562 0.24343217  3.586709e-02       *
Neuroptera:(Intercept)  -1.29578035 0.71874835  7.141508e-02        
Neuroptera:BlockS        3.83945231 0.71466964  7.771895e-08       *
Neuroptera:TreatmentM    1.66188439 0.15126270  4.426118e-28       *
Neuroptera:TreatmentT    0.03774033 0.19429176  8.459835e-01        
Orthoptera:(Intercept)   4.23977003 0.09211991  0.000000e+00       *
Orthoptera:BlockS       -0.82848853 0.14429784  9.383210e-09       *
Orthoptera:TreatmentM   -0.45473616 0.09953271  4.907333e-06       *
Orthoptera:TreatmentT   -0.44268782 0.09916742  8.042674e-06       *
Phasmatodea:(Intercept)  2.31305639 0.20679255  4.807583e-29       *
Phasmatodea:BlockS      -0.75030559 0.29428100  1.078395e-02       *
Phasmatodea:TreatmentM  -0.40546511 0.22821773  7.562435e-02        
Phasmatodea:TreatmentT   0.63965850 0.17838818  3.360887e-04       *
\end{verbatim}

This should read like your standard summary output table. \texttt{Estimate} gives the estimated coefficient for the covariate-spp pairing described by the row name. \texttt{std.error} gives the standard error, \texttt{p.value} the Wald test p-value (I think this is calculated from a Fisher Information Matrix? idk will check that later -- but these p-values are probably not the most robust thing in the world), and \texttt{clarity} gets a star when p < 0.05. It is a bit hard to pick out patterns staring at a table like this, so lets try visualizing it.

\subsection*{1.4 Visualizing model predictions}
\label{sec:org19ac34d}

Lets visualize the predictions rather than the effect estimates themselves, as I think this is a bit easier to look at and interpret and it contains the same information anyways. The one tricky thing is the confidence intervals. I dont think its valid to calculate prediction intervals in this case because of the lognormal tranformation and Poisson weirdness. However I think its okay to report the Wald Confidence intervals (transformed to match the response (i.e. exponentiated)). These will probably be a bit conservative, and reflect uncertainty in the effect estimates, not the predictions, but should do the trick and I doubt anyone except maybe a statistician will take issue. Even then if we just report what we are doing it will be fine.

\begin{verbatim}

## first create some fake data
fake.data <- data.frame(Treatment = c("C", "C", "M", "M", "T", "T"),
                        Block = rep(c("N", "S"), times = 3),
                        Replicate = rep("1"), times = 6)

## generate predictions for the fake data
predictions <- predict(abund_model, newdata = fake.data, se.fit = T)

## make data.frame longform for easier plotting
p.data <- do.call("rbind", replicate(11, fake.data, simplify = FALSE))
p.data$species <- rep(colnames(summary@y), each = 6) ## attach species information

## calculate 95% Wald CIs (these are not prediction intervals!!)
p.data$predictions <- matrix(predictions$fitted.values, ncol = 1)
p.data$ci.lower <- matrix(as.matrix(predictions$fitted.values) - 1.97*as.matrix(predictions$se.fit), ncol = 1)
p.data$ci.upper <- matrix(as.matrix(predictions$fitted.values) + 1.97*as.matrix(predictions$se.fit), ncol = 1)

## transform from the scale of the linear predictors to the response scale (bit of weirdness when transforming the CIs)
p.data$tr.predictions <- abund_model@family@linkinv(p.data$predictions)
p.data$tr.ci.lower <- abund_model@family@linkinv(p.data$ci.lower)
p.data$tr.ci.upper <- abund_model@family@linkinv(p.data$ci.upper)

## generate plots
plotlist <- list()
for (spp in unique(p.data$species)) {

  plotlist[[spp]] <- second_axis(ggplot(data = p.data[p.data$Block == "N" & p.data$species == spp, ],
                                        aes(x = Treatment, y = tr.predictions)) +
                        geom_point(size = 2, color = "#43a2ca") +
                        ylab("predicted abundance") +
                        geom_errorbar(aes(ymin = tr.ci.lower, ymax = tr.ci.upper), size = 1, color = "#43a2ca") +
                        theme_jabo() + ## my custom theme (see /plot_utility.R)
                        theme(legend.position = "none",
                              axis.title = element_blank()) +
                        ggtitle(spp))

}

## align plots
plotlist <- align_plots(plotlist[[1]], plotlist[[2]], plotlist[[3]],
          plotlist[[4]], plotlist[[5]], plotlist[[6]],
          plotlist[[7]], plotlist[[8]], plotlist[[9]],
          plotlist[[10]], plotlist[[11]], align = c("hv"))

## print plots in a grid
plot_grid(plotlist[[1]], plotlist[[2]], plotlist[[3]],
          plotlist[[4]], plotlist[[5]], plotlist[[6]],
          plotlist[[7]], plotlist[[8]], plotlist[[9]],
          plotlist[[10]], plotlist[[11]])
\end{verbatim}

\begin{figure}[htbp]
\centering
\includesvg[width=.9\linewidth]{abundance_estimates}
\caption{Predicted abundance by treatment and species. Lines give 95\% Wald confidence intervals.}
\end{figure}


I don't know the ecology well enough to make a nice interpretation of these. Some of them jump out at me as being intuitive though. For example, the more mammal-dependent orders (Diptera and Acari (ticks?)) decrease when mammals are excluded. Orthoptera also decreases, perhaps because the vegatation becomes less grassy? I am probably making up stories here so I will leave it to the people who know better. I think the Wald Confidence intervals are conservative, so that is probably why the trends appear less clear than the model output table might suggest.

\section*{2. Biomass}
\label{sec:org85c5198}

Okay now lets apply a similar analysis to the biomass data. This is a lot more straightforward since its continuous data and therefore doesn't require a GLM. We can fit a multivariate linear model in base R using the standard \texttt{lm()} function, just providing multiple responses. I don't think its necessary to go over the model, but check out this \href{https://en.wikipedia.org/wiki/Multivariate\_normal\_distribution}{wikipedia page} for more info.

\subsection*{2.1 Loading the data}
\label{sec:org037fd95}

Let's first make sure the data look good.

\begin{verbatim}

uhurub <- read.csv("UHURU_summary_metadata_biomass.csv")

## remove biomass from colnames
colnames(uhurub) <- apply(X = matrix(colnames(uhurub)), MARGIN = 1, FUN = removeWords, stopwords = "biomass")

## inspect data
uhurub[,17:32]

\end{verbatim}

\begin{verbatim}
    Acari Aranae Blattodea Coleoptera Diptera Gatropoda Hemiptera Hymenoptera
1  0.0093 0.1877    0.0000     0.1552  0.6248    0.0000    0.5794      0.3538
2  0.0104 0.1621    0.0000     0.5555  0.0850    0.0000    0.9294      1.0258
3  0.0138 0.1349    0.0129     0.6489  0.6462    0.0205    1.0775      0.5822
4  0.0134 0.0990    0.0000     0.9925  0.4539    0.0000    0.8102      1.0337
5  0.0266 0.3812    0.0148     0.4840  0.4747    0.0000    0.5384      1.0525
6  0.0000 0.0380    0.0000     0.3020  0.1971    0.0000    0.4333      0.4362
7  0.0155 0.0902    0.0000     0.0249  0.0884    0.0000    0.6546      0.1479
8  0.0093 0.2208    0.0000     0.5319  0.0911    0.0000    0.7251      0.1416
9  0.0051 0.1318    0.0000     0.1657  0.1528    0.0000    0.1538      0.2093
10 0.0113 0.0337    0.0000     0.7848  0.3161    0.0156    0.4921      0.8511
11 0.0033 0.1017    0.0000     0.8117  0.3694    0.0000    0.8843      1.2420
12 0.0011 0.5058    0.0000     0.4233  0.4351    0.0000    0.5744      0.3852
13 0.0411 0.1180    0.0000     0.3453  2.6139    0.0000    0.7170      1.0123
14 0.2120 0.3027    0.0000     2.0621  0.3095    0.0000    2.2359      0.9865
15 0.0035 0.1487    0.0000     0.7104  0.5255    0.0951    0.7982      1.0041
16 0.0113 0.2100    0.0000     0.4446  0.5739    0.0041    0.6062      0.8741
17 0.0000 0.1264    0.0000     0.6996  0.2661    0.0227    0.5179      0.9228
18 0.0094 0.7530    0.0000     0.7343  0.8281    0.0000    0.8187      0.3946
   Isopoda Lepidoptera Mantodea Neuroptera Orthoptera Phasmatodea Psocoptera
1   0.0000      0.5288   0.0605     0.0000    3.13460      0.1701      0e+00
2   0.0000      0.8221   0.1874     0.0032    3.11630      0.3257      0e+00
3   0.0000      1.3198   0.0000     0.0000    3.48570      0.7597      0e+00
4   0.0000      1.4595   0.0000     0.0000    0.78910      0.4010      0e+00
5   0.0000      1.2488   0.0000     0.0591    1.47200      0.1498      0e+00
6   0.0000      0.8287   0.0000     0.0000    1.27390      1.0866      1e-04
7   0.0000      0.5231   0.0000     0.0190    3.56973      0.9052      0e+00
8   0.0000      0.3958   0.0000     0.0264    0.66810      0.2190      0e+00
9   0.0000      0.2260   0.0175     0.0145    0.56360      1.1174      0e+00
10  0.0000      0.3388   0.0625     0.1130    1.12850      0.5653      0e+00
11  0.0000      1.5954   0.0255     0.1646    1.74710      0.1313      0e+00
12  0.0000      0.7307   0.1623     0.0043    1.58500      0.3043      0e+00
13  0.0059      1.1682   0.2968     0.0274    1.91100      0.4977      0e+00
14  0.0006      3.1649   0.0176     0.2827    3.50670      0.1780      0e+00
15  0.0000      1.0551   0.0335     0.0149    0.68200      0.5270      0e+00
16  0.1244      0.4573   0.0620     0.0277    0.62610      0.0865      0e+00
17  0.0013      0.5177   0.0678     0.0172    1.06100      0.0568      0e+00
18  0.0030      1.3931   0.1172     0.1172    1.63430      0.3996      0e+00
   Solifugae
1     0.0778
2     0.0000
3     0.0000
4     0.0000
5     0.0000
6     0.0026
7     0.0000
8     0.0000
9     0.0000
10    0.0000
11    0.0000
12    0.0000
13    0.0000
14    0.0000
15    0.0000
16    0.0000
17    0.0000
18    0.0000
\end{verbatim}

Its all there!

\subsection*{2.2 Fitting the model}
\label{sec:orgeb06229}

\begin{verbatim}

biom_model <- lm(cbind(Acari,
                       Aranae,
                       Coleoptera,
                       Diptera,
                       Hemiptera,
                       Hymenoptera,
                       Lepidoptera,
                       Mantodea,
                       Neuroptera,
                       Orthoptera,
                       Phasmatodea) ~ Treatment + Block + Block:Replicate,
                 data = uhurub)

summary(biom_model)

\end{verbatim}

\begin{verbatim}
Response Acari :

Call:
lm(formula = Acari ~ Treatment + Block + Block:Replicate, data = uhurub)

Residuals:
      Min        1Q    Median        3Q       Max 
-0.054967 -0.016633 -0.006958  0.008975  0.157867 

Coefficients:
                   Estimate Std. Error t value Pr(>|t|)
(Intercept)       0.0076500  0.0496701   0.154    0.880
TreatmentM        0.0266167  0.0304166   0.875    0.399
TreatmentT       -0.0115000  0.0304166  -0.378    0.712
BlockS            0.0182000  0.0657073   0.277    0.787
BlockN:Replicate -0.0006000  0.0215078  -0.028    0.978
BlockS:Replicate  0.0008333  0.0215078   0.039    0.970

Residual standard error: 0.05268 on 12 degrees of freedom
Multiple R-squared:  0.1652,	Adjusted R-squared:  -0.1826 
F-statistic: 0.4749 on 5 and 12 DF,  p-value: 0.7882


Response Aranae :

Call:
lm(formula = Aranae ~ Treatment + Block + Block:Replicate, data = uhurub)

Residuals:
     Min       1Q   Median       3Q      Max 
-0.21158 -0.09606 -0.02425  0.05443  0.34547 

Coefficients:
                  Estimate Std. Error t value Pr(>|t|)
(Intercept)       0.089606   0.171977   0.521    0.612
TreatmentM        0.092717   0.105314   0.880    0.396
TreatmentT        0.162267   0.105314   1.541    0.149
BlockS           -0.068444   0.227505  -0.301    0.769
BlockN:Replicate -0.006983   0.074468  -0.094    0.927
BlockS:Replicate  0.074700   0.074468   1.003    0.336

Residual standard error: 0.1824 on 12 degrees of freedom
Multiple R-squared:  0.2781,	Adjusted R-squared:  -0.02263 
F-statistic: 0.9248 on 5 and 12 DF,  p-value: 0.4981


Response Coleoptera :

Call:
lm(formula = Coleoptera ~ Treatment + Block + Block:Replicate, 
    data = uhurub)

Residuals:
     Min       1Q   Median       3Q      Max 
-0.30962 -0.23338 -0.10087  0.07325  1.02933 

Coefficients:
                 Estimate Std. Error t value Pr(>|t|)
(Intercept)       0.49494    0.39939   1.239    0.239
TreatmentM        0.39958    0.24458   1.634    0.128
TreatmentT        0.03955    0.24458   0.162    0.874
BlockS            0.18534    0.52835   0.351    0.732
BlockN:Replicate -0.10618    0.17294  -0.614    0.551
BlockS:Replicate -0.02355    0.17294  -0.136    0.894

Residual standard error: 0.4236 on 12 degrees of freedom
Multiple R-squared:  0.359,	Adjusted R-squared:  0.09186 
F-statistic: 1.344 on 5 and 12 DF,  p-value: 0.3112


Response Diptera :

Call:
lm(formula = Diptera ~ Treatment + Block + Block:Replicate, data = uhurub)

Residuals:
     Min       1Q   Median       3Q      Max 
-0.56135 -0.15770 -0.12841  0.07426  1.64520 

Coefficients:
                 Estimate Std. Error t value Pr(>|t|)
(Intercept)       0.92953    0.53137   1.749    0.106
TreatmentM       -0.51253    0.32540  -1.575    0.141
TreatmentT       -0.31437    0.32540  -0.966    0.353
BlockS           -0.14333    0.70293  -0.204    0.842
BlockN:Replicate -0.17062    0.23009  -0.742    0.473
BlockS:Replicate  0.09125    0.23009   0.397    0.699

Residual standard error: 0.5636 on 12 degrees of freedom
Multiple R-squared:  0.3056,	Adjusted R-squared:  0.01624 
F-statistic: 1.056 on 5 and 12 DF,  p-value: 0.4303


Response Hemiptera :

Call:
lm(formula = Hemiptera ~ Treatment + Block + Block:Replicate, 
    data = uhurub)

Residuals:
     Min       1Q   Median       3Q      Max 
-0.54943 -0.18088 -0.11679  0.07509  1.16723 

Coefficients:
                  Estimate Std. Error t value Pr(>|t|)  
(Intercept)       0.897350   0.412946   2.173   0.0505 .
TreatmentM        0.328583   0.252877   1.299   0.2182  
TreatmentT       -0.000600   0.252877  -0.002   0.9981  
BlockS           -0.154600   0.546277  -0.283   0.7820  
BlockN:Replicate -0.175467   0.178811  -0.981   0.3458  
BlockS:Replicate -0.001333   0.178811  -0.007   0.9942  
---
Signif. codes:  0 ‘***’ 0.001 ‘**’ 0.01 ‘*’ 0.05 ‘.’ 0.1 ‘ ’ 1

Residual standard error: 0.438 on 12 degrees of freedom
Multiple R-squared:  0.2546,	Adjusted R-squared:  -0.05603 
F-statistic: 0.8196 on 5 and 12 DF,  p-value: 0.5586


Response Hymenoptera :

Call:
lm(formula = Hymenoptera ~ Treatment + Block + Block:Replicate, 
    data = uhurub)

Residuals:
     Min       1Q   Median       3Q      Max 
-0.45276 -0.14675  0.01103  0.13734  0.47098 

Coefficients:
                 Estimate Std. Error t value Pr(>|t|)   
(Intercept)       1.05039    0.27572   3.810  0.00249 **
TreatmentM        0.18305    0.16884   1.084  0.29960   
TreatmentT       -0.21022    0.16884  -1.245  0.23687   
BlockS           -0.09321    0.36474  -0.256  0.80262   
BlockN:Replicate -0.24383    0.11939  -2.042  0.06373 . 
BlockS:Replicate -0.04780    0.11939  -0.400  0.69592   
---
Signif. codes:  0 ‘***’ 0.001 ‘**’ 0.01 ‘*’ 0.05 ‘.’ 0.1 ‘ ’ 1

Residual standard error: 0.2924 on 12 degrees of freedom
Multiple R-squared:  0.5466,	Adjusted R-squared:  0.3576 
F-statistic: 2.893 on 5 and 12 DF,  p-value: 0.0611


Response Lepidoptera :

Call:
lm(formula = Lepidoptera ~ Treatment + Block + Block:Replicate, 
    data = uhurub)

Residuals:
     Min       1Q   Median       3Q      Max 
-0.89409 -0.41352  0.01633  0.23931  1.70364 

Coefficients:
                 Estimate Std. Error t value Pr(>|t|)
(Intercept)       1.08407    0.68097   1.592    0.137
TreatmentM        0.54483    0.41701   1.307    0.216
TreatmentT        0.17962    0.41701   0.431    0.674
BlockS           -0.06871    0.90084  -0.076    0.940
BlockN:Replicate -0.25430    0.29487  -0.862    0.405
BlockS:Replicate -0.04947    0.29487  -0.168    0.870

Residual standard error: 0.7223 on 12 degrees of freedom
Multiple R-squared:  0.2282,	Adjusted R-squared:  -0.09341 
F-statistic: 0.7095 on 5 and 12 DF,  p-value: 0.6277


Response Mantodea :

Call:
lm(formula = Mantodea ~ Treatment + Block + Block:Replicate, 
    data = uhurub)

Residuals:
     Min       1Q   Median       3Q      Max 
-0.06433 -0.05041 -0.02019  0.02806  0.18429 

Coefficients:
                 Estimate Std. Error t value Pr(>|t|)
(Intercept)       0.12489    0.07814   1.598    0.136
TreatmentM       -0.03058    0.04785  -0.639    0.535
TreatmentT       -0.02522    0.04785  -0.527    0.608
BlockS           -0.01128    0.10337  -0.109    0.915
BlockN:Replicate -0.03840    0.03384  -1.135    0.279
BlockS:Replicate -0.00055    0.03384  -0.016    0.987

Residual standard error: 0.08288 on 12 degrees of freedom
Multiple R-squared:  0.2715,	Adjusted R-squared:  -0.03201 
F-statistic: 0.8946 on 5 and 12 DF,  p-value: 0.5149


Response Neuroptera :

Call:
lm(formula = Neuroptera ~ Treatment + Block + Block:Replicate, 
    data = uhurub)

Residuals:
      Min        1Q    Median        3Q       Max 
-0.090967 -0.039617  0.007767  0.016333  0.154567 

Coefficients:
                  Estimate Std. Error t value Pr(>|t|)
(Intercept)      -0.023650   0.062975  -0.376    0.714
TreatmentM        0.061017   0.038564   1.582    0.140
TreatmentT       -0.006033   0.038564  -0.156    0.878
BlockS            0.130700   0.083308   1.569    0.143
BlockN:Replicate  0.009450   0.027269   0.347    0.735
BlockS:Replicate -0.019967   0.027269  -0.732    0.478

Residual standard error: 0.06679 on 12 degrees of freedom
Multiple R-squared:  0.4436,	Adjusted R-squared:  0.2117 
F-statistic: 1.913 on 5 and 12 DF,  p-value: 0.1657


Response Orthoptera :

Call:
lm(formula = Orthoptera ~ Treatment + Block + Block:Replicate, 
    data = uhurub)

Residuals:
    Min      1Q  Median      3Q     Max 
-1.3036 -0.6587 -0.2613  0.2679  2.2996 

Coefficients:
                 Estimate Std. Error t value Pr(>|t|)   
(Intercept)       3.73775    1.03796   3.601  0.00364 **
TreatmentM        0.06869    0.63562   0.108  0.91572   
TreatmentT       -0.32242    0.63562  -0.507  0.62117   
BlockS           -1.73103    1.37309  -1.261  0.23138   
BlockN:Replicate -0.82253    0.44945  -1.830  0.09217 . 
BlockS:Replicate -0.18987    0.44945  -0.422  0.68017   
---
Signif. codes:  0 ‘***’ 0.001 ‘**’ 0.01 ‘*’ 0.05 ‘.’ 0.1 ‘ ’ 1

Residual standard error: 1.101 on 12 degrees of freedom
Multiple R-squared:  0.2842,	Adjusted R-squared:  -0.01404 
F-statistic: 0.9529 on 5 and 12 DF,  p-value: 0.4828


Response Phasmatodea :

Call:
lm(formula = Phasmatodea ~ Treatment + Block + Block:Replicate, 
    data = uhurub)

Residuals:
     Min       1Q   Median       3Q      Max 
-0.33847 -0.15526  0.04995  0.16139  0.25483 

Coefficients:
                 Estimate Std. Error t value Pr(>|t|)  
(Intercept)       0.24160    0.20477   1.180   0.2609  
TreatmentM       -0.26087    0.12539  -2.080   0.0596 .
TreatmentT        0.26147    0.12539   2.085   0.0591 .
BlockS            0.21603    0.27088   0.798   0.4406  
BlockN:Replicate  0.16435    0.08867   1.854   0.0885 .
BlockS:Replicate -0.07633    0.08867  -0.861   0.4062  
---
Signif. codes:  0 ‘***’ 0.001 ‘**’ 0.01 ‘*’ 0.05 ‘.’ 0.1 ‘ ’ 1

Residual standard error: 0.2172 on 12 degrees of freedom
Multiple R-squared:  0.7018,	Adjusted R-squared:  0.5776 
F-statistic: 5.649 on 5 and 12 DF,  p-value: 0.006625
\end{verbatim}

Looks like there aren't any clear trends by treatment. Not sure how to interpret this in relation to the abundance trends, but one issue that comes to mind is the influence a single large individual (a grasshopper for example) can have on the biomass of some of these orders. This shouldn't be the case for things like Hymenoptera which were mostly ants though. Because there are no trends, I won't bother plotting the predictions this time, though we could later on.


\section*{3. Diversity}
\label{sec:org33278a9}

Finally we can look at the effect of treatment on diversity. This is the simplest one because there is just a single response variable and we can just use a plain old linear model. Let's do it all in one step:

\begin{verbatim}
uhurua <- read.csv("UHURU_summary_metadata_abundance.csv",row.names = 1, header=TRUE)

colnames(uhurua)[33] <- "Shan_div"

summary(lm(Shan_div ~ Treatment + Block + Block:Replicate, data = uhurua))

\end{verbatim}

\begin{verbatim}

Call:
lm(formula = Shan_div ~ Treatment + Block + Block:Replicate, 
    data = uhurua)

Residuals:
     Min       1Q   Median       3Q      Max 
-0.53388 -0.20524  0.00823  0.12047  0.53958 

Coefficients:
                 Estimate Std. Error t value Pr(>|t|)    
(Intercept)       3.67198    0.32434  11.321 9.22e-08 ***
TreatmentM        0.07357    0.19862   0.370   0.7175    
TreatmentT       -0.01925    0.19862  -0.097   0.9244    
BlockS           -0.99174    0.42906  -2.311   0.0394 *  
BlockN:Replicate -0.07148    0.14044  -0.509   0.6200    
BlockS:Replicate  0.09860    0.14044   0.702   0.4960    
---
Signif. codes:  0 ‘***’ 0.001 ‘**’ 0.01 ‘*’ 0.05 ‘.’ 0.1 ‘ ’ 1

Residual standard error: 0.344 on 12 degrees of freedom
Multiple R-squared:  0.5882,	Adjusted R-squared:  0.4166 
F-statistic: 3.428 on 5 and 12 DF,  p-value: 0.03731
\end{verbatim}

No clear trend by treatment. Once again I wont bother plotting these results.
\end{document}
